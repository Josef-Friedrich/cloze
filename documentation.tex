%!TEX program = lualatex
\documentclass{article}
\usepackage{listings}
\usepackage{luacloze}

\definecolor{codebackground}{rgb}{0.95,0.95,0.95}

\lstset{
  basicstyle=\ttfamily,
  language=TeX,
  frame=single,
  backgroundcolor=\color{codebackground},
}

\newcommand{\result}[1]{\noindent\fbox{\begin{minipage}{12cm}#1\end{minipage}}}

\begin{document}

\section{Basic usage}

\begin{lstlisting}
\usepackage{luacloze}
\end{lstlisting}

\begin{lstlisting}
Lorem ipsum \cloze{dolor} sit amet, consectetur
adipisicing elit, sed do eiusmod tempor
incididunt ut labore et dolore magna aliqua.
\end{lstlisting}

\result{Lorem ipsum \cloze{dolor} sit amet, consectetur adipisicing
elit, sed do eiusmod tempor incididunt ut labore et dolore magna
aliqua.}

\subsection{Hide cloze text}

\begin{lstlisting}
\setcloze{hide}
\end{lstlisting}

\setcloze{hide}

\result{Lorem ipsum \cloze{dolor} sit amet, consectetur adipisicing
elit, sed do eiusmod tempor incididunt ut labore et dolore magna
aliqua.}

\subsection{Show cloze text}

\begin{lstlisting}
\setcloze{show}
\end{lstlisting}

\setcloze{show}

\result{Lorem ipsum \cloze{dolor} sit amet, consectetur adipisicing
elit, sed do eiusmod tempor incididunt ut labore et dolore magna
aliqua.}


%-----------------------------------------------------------------------
% Different modes
%-----------------------------------------------------------------------

\section{Different modes}

At present there are four different modes to generate cloze text.

%%
% basic
%%

\subsection{Basic (basic)}

\begin{lstlisting}
\cloze{Lorem ipsum}
\end{lstlisting}

%%
% fix
%%

\subsection{Fixed (fix)}

Creates a cloze on one line with a fixed length.

\begin{lstlisting}
\clozefix{Lorem ipsum}
\end{lstlisting}

%%
% end
%%

\subsection{End (end)}

This command forces a line break and creates a cloze line that extends
to the end of the text line.

\begin{lstlisting}
\clozeend{Lorem ipsum}
\end{lstlisting}

%%
% par
%%

\subsection{Paragraph (par)}

\begin{lstlisting}
\begin{clozepar}
Lorem ipsum
\end{clozepar}
\end{lstlisting}

%-----------------------------------------------------------------------
% Cloze line thickness
%-----------------------------------------------------------------------

\subsection{Cloze line thickness}

%%
% 0.5pt
%%

\begin{lstlisting}
\setcloze{thickness=0.5pt}
\end{lstlisting}

\setcloze{thickness=0.5pt}

\result{Lorem \cloze{ipsum dolor} sit amet.}

%%
% 1mm
%%

\begin{lstlisting}
\setcloze{thickness=1mm}
\end{lstlisting}

\setcloze{thickness=1mm}

\result{Lorem \cloze{ipsum dolor} sit amet.}

%%
% 0.15em
%%

\begin{lstlisting}
\setcloze{thickness=0.15em}
\end{lstlisting}

\setcloze{thickness=0.15em}

\result{Lorem \cloze{ipsum dolor} sit amet.}

%-----------------------------------------------------------------------
% Descender of the cloze line
%-----------------------------------------------------------------------

\subsection{Descender of the cloze line}

%%
% 0pt
%%

\begin{lstlisting}
\setcloze{descender=0pt}
\end{lstlisting}

\setcloze{descender=0pt,thickness=0.4pt}

\result{Lorem \cloze{ipsum dolor} sit amet.}

%%
% 3pt
%%

\begin{lstlisting}
\setcloze{descender=3pt}
\end{lstlisting}

\setcloze{descender=3pt}

\result{Lorem \cloze{ipsum dolor} sit amet.}


%%
% -4pt
%%

\begin{lstlisting}
\setcloze{descender=-4pt}
\end{lstlisting}

\setcloze{descender=-4pt}

\result{Lorem \cloze{ipsum dolor} sit amet.}


\renewcommand{\b}[1]{\fbox{#1}}

\begin{itemize}
\item B: Begin whatsit user\_definded
\item cl: Color for cloze line.
\item R: Rule
\item cr: Color reset (Black).
\item K: kern
\item a,b,c: Cloze text
\item E: Begin whatsit user\_definded
\end{itemize}

\b{E} \b{cl} \b{R} \b{cr} \b{K} \b{a} \b{b} \b{c} \b{K} \b{B}

\end{document}