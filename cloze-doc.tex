\documentclass{ltxdoc}
\usepackage[show]{cloze}
\usepackage{graphicx}
\usepackage{fontspec}
\usepackage{paralist}
\usepackage{forest}
\usepackage{multicol}
\usepackage{amsmath}
\usepackage{tikz}
\usetikzlibrary{positioning}

%%
% Make underscore an "other" character
%%
\catcode`\_=12

%%
% titlesec
%%

\usepackage{titlesec}
\titleformat{\paragraph}[hang]{%
  \normalfont\normalsize\bfseries%
  }{\theparagraph}{1em}{}
\titleformat{\subparagraph}[hang]{%
  \normalfont\small\bfseries%
  }{\thesubparagraph}{1em}{}
  \titlespacing*{\subparagraph}{0pt}{*1}{0pt}

%%
% hyperref
%%

\usepackage[
  colorlinks=true,
  linkcolor=red,
  filecolor=red,
  urlcolor=red,
]{hyperref}

%%
% mdframed
%%

\usepackage{mdframed}
\definecolor{TmpGray}{gray}{0.95}
% f2d6aa 242 214 170
\definecolor{TmpYellow}{rgb}{0.949019608, 0.839215686, 0.666666667}
\newmdenv[
  backgroundcolor=TmpYellow!3,
  linecolor=TmpYellow!70
]{TmpExample}
\surroundwithmdframed[backgroundcolor=TmpGray]{verbatim}

%%
% minted
%%

\usepackage{minted}
\usemintedstyle{colorful}
\BeforeBeginEnvironment{minted}{\begin{mdframed}[
  backgroundcolor=TmpGray!60,
  linecolor=gray!40,
]}
\AfterEndEnvironment{minted}{\end{mdframed}}
\setminted{
  breaklines=true,
  fontsize=\footnotesize,
}

\MakeShortVerb{\|}

%%
% Temporary local macros
%%

\newcommand{\TmpExpDesc}[1]{(|#1|)}

\newcommand{\TmpDesc}[1]{%
  \hfill%
  \TmpExpDesc{#1}%
  \par%
}

\def\tt#1{%
  \texttt{#1}%
}

\def\TmpSecRef#1{%
  (\rightarrow\ \ref{#1})%
}

\def\TmpOption#1#2{%
  \tt{[#1=}\meta{#2}\tt{]}%
}

\def\TmpHide{%
  \noindent%
  \texttt{\string\clozehide}:%
  \clozehide%
}

\def\TmpShow{%
  \noindent%
  \texttt{\string\clozeshow}:%
  \clozeshow%
}

\def\TmpReduceVspace{%
  \vspace{-0.2cm}%
}

\EnableCrossrefs
\CodelineIndex
\RecordChanges
\begin{document}

%%%%%%%%%%%%%%%%%%%%%%%%%%%%%%%%%%%%%%%%%%%%%%%%%%%%%%%%%%%%%%%%%%%%%%%%
%
%%%%%%%%%%%%%%%%%%%%%%%%%%%%%%%%%%%%%%%%%%%%%%%%%%%%%%%%%%%%%%%%%%%%%%%%

\providecommand*{\url}{\texttt}
\GetFileInfo{cloze.dtx}
\title{The \cloze{cloze} package%
  \thanks{Many thanks to Robert-Michael Huber for his advice
and to Paul Isambert for his article \emph{"Three things you can do with
Lua\TeX{} that would be extremely painful otherwise"} in TUGboat, Volume
31 (2010), No. 3. This article helped a lot to write this package.}%
}
\author{%
  Josef Friedrich\\%
  \url{josef@friedrich.rocks}\\%
  \href{https://github.com/Josef-Friedrich/cloze}
  {github.com/Josef-Friedrich/cloze}%
}
\date{v2.0.0~from 2025/06/19}

\maketitle\vfill

\pagebreak

\setcounter{secnumdepth}{5}
\setcounter{tocdepth}{5}
\tableofcontents

%-----------------------------------------------------------------------
% Introduction
%-----------------------------------------------------------------------

\pagebreak
\section{Introduction}

\emph{cloze} is a plain \TeX{} or a \LaTeX{} package to generate cloze
texts. It uses the capabilities of the modern \TeX{} engine
\emph{Lua\TeX}. Therefore, you must use Lua\TeX{} or Lua\LaTeX{} to
create documents containing gaps.

\begin{center}
\begin{minipage}{0.4\linewidth}
\begin{verbatim}
lualatex cloze-text.tex
\end{verbatim}
\end{minipage}
or
\begin{minipage}{0.4\linewidth}
\begin{verbatim}
luatex cloze-text.tex
\end{verbatim}
\end{minipage}
\end{center}

\noindent
The main feature of the package is that the formatting doesn't change
when using the |hide| and |show| \TmpSecRef{sec:option-hide} options.

\newcommand{\clozelorem}{%
Lorem ipsum \cloze{dolor sit} amet, consectetur \cloze{adipisicing}
elit, sed do eiusmod tempor incididunt ut labore et \cloze{dolore magna}
aliqua. Ut enim ad minim veniam, quis nostrud \cloze{exercitation}
ullamco laboris nisi ut \cloze{aliquip} ex ea commodo consequat.%
}

\begin{TmpExample}
\clozelorem
\end{TmpExample}

\clozeset{hide}

\noindent
The command |\clozeset{hide}| only shows gaps. When you put both texts
on top of each other you will see that they perfectly match.

\begin{TmpExample}
\clozelorem
\end{TmpExample}

\clozeset{show}

%-----------------------------------------------------------------------
% Usage
%-----------------------------------------------------------------------

\section{Usage}

%-----------------------------------------------------------------------
%
%-----------------------------------------------------------------------

\subsection{Interfaces}

%%
%
%%

\subsubsection{The plain \TeX{} interface}

\begin{minted}{tex}
\input cloze.tex
\clozeset{margin=1cm}
\clozeshow
Lorem \cloze{ipsum} dolor.
\bye
\end{minted}

%%
%
%%

\subsubsection{The \LaTeX{} interface}

\begin{minted}{latex}
\documentclass{article}
\usepackage[show,margin=1cm]{cloze}
\begin{document}
Lorem \cloze{ipsum} dolor.
\end{document}
\end{minted}

\subsection{The commands and environments}

There are the commands
\cmd{\cloze},
\cmd{\clozefix},
\cmd{\clozefil},
\cmd{\clozenol},
\cmd{\clozeparcapture},
\cmd{\clozestrike} and the environments
|clozepar| and
|clozebox|
to generate cloze texts.

%%
% \cloze
%%

\subsubsection{\cmd{\cloze}}
\label{sec:command-cloze}

\DescribeMacro{\cloze} \cmd{\cloze}\oarg{options}\marg{some text}: The
command \cmd{\cloze} is similar to a command that offers the possibility
to underline the texts. \cmd{\cloze} does not prevent line breaks. The
width of a gap depends on the number of letters and the font used.
The only option which affects the widths of a gap is the option
|margin| \TmpSecRef{sec:option-margin}.

\begin{TmpExample}
Lorem ipsum \cloze{dolor} sit amet, \cloze{consectetur} adipisicing
elit, sed do eiusmod tempor incididunt ut labore et dolore
\cloze{magna aliqua. Ut enim ad minim veniam, quis nostrud exercitation
ullamco laboris nisi} ut aliquip ex ea commodo consequat.
\end{TmpExample}

\noindent It is possible to convert a complete paragraph into a `gap'.
But don't forget: There is a special environment for this: \tt{clozepar}
\TmpSecRef{sec:command-clozepar}.

\begin{TmpExample}
\cloze{Lorem ipsum dolor sit amet, consectetur adipisicing elit, sed do
eiusmod tempor incididunt ut labore et dolore magna aliqua. Ut enim ad
minim veniam, quis nostrud exercitation ullamco laboris nisi ut aliquip
ex ea commodo consequat.}
\end{TmpExample}

\noindent The command \cmd{\cloze} doesn’t (better shouldn’t)  change
the behavior of the hyphenation. Let’s try some long German words (Known
issue: Sometimes you have to insert manual hyphenation points (|\-|). I
don’t know why. Solutions and suggestions are very welcome.):

\begin{TmpExample}
es
\cloze{Te\-le\-kom\-mu\-ni\-ka\-tions\-ü\-ber\-wach\-ung}
geht
\cloze{Un\-ter\-neh\-mens\-steu\-er\-fort\-ent\-wick\-lungs\-ge\-setz}
\cloze{Ab\-teil\-ungs\-lei\-ter\-in}
\cloze{Ober\-kom\-mi\-sar\-in}
auch
\cloze{Fil\-lial\-lei\-ter\-in}
kurz.
\end{TmpExample}

%%
% \clozesetfont
%%

\subsubsection{\cmd{\clozesetfont}}
\label{sec:command-clozesetfont}
\label{sec:command-clozefont}

\DescribeMacro{\clozesetfont}
The gap font can be changed by using the command
\cmd{\clozesetfont}. \tt{\string\cloze\-set\-font} redefines the command
\cmd{\clozefont} which contains the font definition.
Thus, the command \tt{\string\clozesetfont\string{\string\Large\string}}
has the same effect as
|\def\clozefont{\Large}|.

\clozesetfont{\Large}

\begin{TmpExample}
Excepteur \cloze{sint} occaecat \cloze{cupidatat} non proident.
\end{TmpExample}

\noindent Please do not put any color definitions in
\cmd{\clozesetfont}, as it won't work. Use the option
|text_color| instead \TmpSecRef{sec:option-text-color}.

|\clozesetfont{\ttfamily\normalsize}| changes the gap text for example
into a normal sized typewriter font.

\clozesetfont{\ttfamily\normalsize}

\begin{TmpExample}
Excepteur \cloze{sint} occaecat \cloze{cupidatat} non proident.
\end{TmpExample}

\clozesetfont{\itshape}

\noindent
If you use clozes in math mode you have to overwrite the default cloze
font (|\clozesetfont{\itshape}|) of the package, because \cmd{\itshape}
is not available in math mode. |\clozesetfont{}| will do the
trick.

%%
% \clozefix
%%

\subsubsection{\cmd{\clozefix}}
\label{sec:command-clozefix}

\DescribeMacro{\clozefix} \cmd{\clozefix}\oarg{options}\marg{some text}:
The command \cmd{\clozefix} creates gaps with a fixed width. The
clozes are default concering the width \tt{\ClozeGetOption{width}}.

\begin{TmpExample}
\noindent Lorem ipsum dolor sit amet:
\begin{compactenum}
\item \clozefix[width=5cm]{consectetur}
\item \clozefix[width=5cm]{adipisicing}
\item \clozefix[width=5cm]{elit}
\end{compactenum}
sed do eiusmod.
\end{TmpExample}

\noindent
Gaps with a fixed width are much harder to solve.

\begin{TmpExample}
Lorem ipsum dolor \clozefix[align=center,width=3cm]{sit} amet,
\clozefix[align=center,width=3cm]{consectetur} adipisicing elit, sed do
eiusmod tempor incididunt \clozefix[align=center,width=3cm]{ut} labore
et dolore magna aliqua.
\end{TmpExample}

\noindent
Using the option |align| you can make nice tabulars like this:

\begin{TmpExample}
\begin{tabular}{p{5cm}p{4cm}}
\raggedleft Composer & Life span \\

\clozefix[width=5cm,align=right]{Joseph Haydn} &
\clozefix{1723-1809} \\

\clozefix[width=5cm,align=right]{Wolfgang Amadeus Mozart} &
\clozefix{1756-1791} \\

\clozefix[width=5cm,align=right]{Ludwig van Beethoven} &
\clozefix{1770-1827} \\
\end{tabular}
\end{TmpExample}

%%
% \clozenol
%%

\subsubsection{\cmd{\clozenol}}
\label{sec:command-clozenol}
\DescribeMacro{\clozenol} \cmd{\clozenol}\oarg{options}\marg{some text}:
The macro name |clozenol| stands for \emph{“cloze no line”}. As the
the name suggests this macro typesets cloze texts without a line.
\cmd{\clozenol} is a convenient abbreviation for
|\cloze[thickness=0pt]{text}|.

\begin{minted}{latex}
Lorem \clozenol{ipsum dolor} sit amet.
\end{minted}

\begin{TmpExample}
Lorem \clozenol{ipsum dolor} sit amet.
\end{TmpExample}

\begin{minted}{latex}
Lorem \clozenol[text_color=green]{ipsum dolor} sit amet.
\end{minted}

\begin{TmpExample}
Lorem \clozenol[text_color=green]{ipsum dolor} sit amet.
\end{TmpExample}

\noindent
The next examples are showing that \cmd{\clozenol} behaves exactly as
\cmd{\clozenol} with the option |thickness=0pt|
(|\cloze[thickness=0pt]|) set: The text layout doesn’t change if we are
hiding the gaps and the hidden text is not really hidden. It is removed.
It can not be copied.

\begin{mdframed}[backgroundcolor=gray!40]
Lorem ipsum \clozenol{dolor sit amet, consetetur sadipscing elitr, sed
diam nonumy eirmod tempor invidunt ut labore et dolore magna aliquyam
erat, sed diam voluptua} sit amet.
\end{mdframed}

\noindent
Now hide the text

\clozehide

\begin{mdframed}[backgroundcolor=gray!40]
Lorem ipsum \clozenol{dolor sit amet, consetetur sadipscing elitr, sed
diam nonumy eirmod tempor invidunt ut labore et dolore magna aliquyam
erat, sed diam voluptua} sit amet.
\end{mdframed}

\clozeshow

%%
% \clozefil
%%

\subsubsection{\cmd{\clozefil}}
\label{sec:command-clozefil}

\DescribeMacro{\clozefil} \cmd{\clozefil}\oarg{options}\marg{some text}:
The name of the command is inspired by \cmd{\hfil}, \cmd{\hfill}, and
\cmd{\hfilll}. \cmd{\clozefil} fills out all available horizontal
space with a line. The macros \cmd{\clozefill} and \cmd{\clozefilll}
doesn’t exist, only \cmd{\clozefil} (with one |l|) does.

\def\TmpExampleClozefil{
Complete the sentences below to share more about yourself:
\bigskip

\noindent
I am someone who love \clozefil{to write cloze worksheets in \TeX}.

\noindent
I am someone who hate \clozefil{to write cloze worksheets in MS Word}.

\noindent
I am someone who can \clozefil{write cloze worksheets in \TeX}.

\noindent
I am someone who can’t \clozefil{write cloze worksheets in MS Word}.
}

\bigskip

\TmpHide
\begin{TmpExample}
\TmpExampleClozefil
\end{TmpExample}

\TmpShow
\begin{TmpExample}
\TmpExampleClozefil
\end{TmpExample}

%%
% \clozeextend
%%

\subsubsection{\cmd{\clozeextend}}
\label{sec:command-clozeextend}

\DescribeMacro{\clozeextend} \cmd{\clozeextend}\oarg{spaces}: The
command |\clozeextend| adds some invisible placeholders to extend some
cloze texts with blank space. Keep in mind that there is the option
\tt{min_lines} \TmpSecRef{sec:option-min-lines}, if you want to extend
some cloze paragraphs.

Corresponding options:

\begin{description}
\item[extension\_count]: The number of extension units.

\item[extension\_height]: The height of one extension unit (default: 2ex).

\item[extension\_width]: The width of one extension unit (default: 1em).
\end{description}

\begin{minted}{latex}
\begin{itemize}
\item \clozefil{Lorem ipsum dolor sit amet, consectetur adipisicing
elit, sed do eiusmod tempor incididunt ut labore et dolore magna
aliqua.}
\item \clozefil{Ut enim ad minim veniam \clozeextend[20]}
\item \clozefil{quis nostrud \clozeextend[20]}
\end{itemize}
\end{minted}

\begin{TmpExample}
\begin{itemize}
\item \clozefil{Lorem ipsum dolor sit amet, consectetur adipisicing
elit, sed do eiusmod tempor incididunt ut labore et dolore magna
aliqua.}
\item \clozefil{Ut enim ad minim veniam \clozeextend[20]}
\item \clozefil{quis nostrud \clozeextend[20]}
\end{itemize}
\end{TmpExample}

%%
% clozepar
%%

\subsubsection{\tt{clozepar}}
\label{sec:command-clozepar}

\DescribeEnv{clozepar} |\begin{clozepar}|\oarg{options} \dots
\textit{some text} \dots |\end{clozepar}|: The environment \tt{clozepar}
transforms a complete paragraph into a cloze text. The options |align|,
|margin| and |width| have no effect on this environment.

\begin{TmpExample}
Lorem ipsum dolor sit amet, consectetur adipisicing elit ullamco laboris
nisi.

\begin{clozepar}
Ut enim ad minim veniam, quis nostrud exercitation ullamco laboris nisi
ut aliquip ex ea commodo consequat. Duis aute irure dolor in
reprehenderit in voluptate velit esse cillum.
\end{clozepar}

Excepteur sint occaecat cupidatat non proident.
\end{TmpExample}

%%
% \clozeparplain
%%

\subsubsection{\cmd{\clozeparplain}}
\label{sec:command-clozeparplain}
\DescribeMacro{\clozeparplain} \cmd{\clozeparplain}: The command
\cmd{\clozeparplain} is the macro version of the environment |clozepar|
\TmpSecRef{sec:command-clozepar}. It is intended to provide cloze
support on full paragraphs in plain Lua\TeX.

%%
% \clozeparcapture
%%

\subsubsection{\cmd{\clozeparcapture}}
\label{sec:command-clozeparcapture}
\DescribeMacro{\clozeparcapture} \cmd{\clozeparcapture}: The command is
defined to capture all text until the next paragraph (|\par|):
\mintinline{latex}{\def\clozeparcapture#1\par{...}}. In some cases (see
example below) you have to set the paragraph end manually by inserting a
|\par|.

\begin{minted}{latex}
\begin{enumerate}
\item Solid state drives (SSD) ...

\clozeparcapture
They do not need to get up to speed to work properly / No latency;
...
\par

\end{enumerate}
\end{minted}

\TmpReduceVspace

% https://www.bitsofbytes.co/exam-questions---memory--storage.html
\begin{TmpExample}
\begin{enumerate}
\item Solid state drives (SSD) are replacing hard disc drives (HDD) in
some computers. Give some reasons why this is happening.

\clozeparcapture
They do not need to get up to speed to work properly / No latency;
Less power consumption / More energy efficient;
Run cooler;
Run quieter;
Data access is faster;
Occupies less physical space - more compact;
Lighter, so suitable for portable computer / laptop;
No moving parts so more reliable / durable in a portable computer / laptop.
\par
\end{enumerate}
\end{TmpExample}

\noindent
It is often sufficient to insert an empty line.

\begin{minted}{latex}
Explain three ways that RAM is different to ROM.

1:

\clozeparcapture RAM is Volatile, ROM is non-volatile

2:
\end{minted}

\TmpReduceVspace
% https://www.bitsofbytes.co/exam-questions---memory--storage.html
\begin{TmpExample}
\noindent
Explain three ways that RAM is different to ROM.

\def\TmpWay#1{\medskip\noindent\textbf{#1:}\medskip}

\TmpWay{1}

\clozeparcapture RAM is Volatile, ROM is non-volatile

\TmpWay{2}

\clozeparcapture RAM is temporary, ROM is (semi) permanent

\TmpWay{3}

\clozeparcapture RAM normally has a larger capacity than ROM \par
\end{TmpExample}

%%
% clozebox
%%

\subsubsection{\tt{clozebox}}
\label{sec:command-clozebox}

\DescribeEnv{clozebox} |\begin{clozebox}|*\oarg{options} \dots
\textit{some text} \dots |\end{clozebox}|: The environment \tt{clozebox}
surrounds a text with a box. The starred version omits the line around
the box. Use the options |box_height| \TmpSecRef{sec:option-box-height}
and |box_width| \TmpSecRef{sec:option-box-width} to specify the dimensions
of the box. By default the width of the box is |\linewidth|. The height
of the box is determined by the amount of text. This environment is
realized by a combination of the |minipage| environment surrounded by a
\cmd{\fbox}. For the cloze text the macro \cmd{\clozenol} is reused. New
paragraphs are not allowed inside a cloze box. Use two backslashes
multiple times |\\| instead.

\begin{minted}{latex}
\begin{clozebox}
Lorem ipsum dolor sit amet, consectetur adipisicing elit, sed do eiusmod
tempor incididunt ut labore et dolore magna aliqua.
\end{clozebox}
\end{minted}

\begin{multicols}{2}
\TmpHide

\begin{TmpExample}
\begin{clozebox}
Lorem ipsum dolor sit amet, consectetur adipisicing elit, sed do eiusmod
tempor incididunt ut labore et dolore magna aliqua.
\end{clozebox}
\end{TmpExample}

\columnbreak

\TmpShow

\begin{TmpExample}
\begin{clozebox}
Lorem ipsum dolor sit amet, consectetur adipisicing elit, sed do eiusmod
tempor incididunt ut labore et dolore magna aliqua.
\end{clozebox}
\end{TmpExample}

\end{multicols}

\noindent
Like with all cloze macros and environments the hidden text vanishes
from the rendered file. The starred version omits the line around the
box:

% starred

\begin{multicols}{2}
\begin{minted}{latex}
\begin{clozebox}*
Lorem ...
\end{clozebox}
\end{minted}

\columnbreak

\begin{TmpExample}
\begin{center}
\begin{clozebox}*
Lorem ipsum dolor sit amet, consectetur adipisicing elit, sed do eiusmod
tempor incididunt ut labore et dolore magna aliqua.
\end{clozebox}
\end{center}
\end{TmpExample}
\end{multicols}

%%
% clozespace
%%

\subsubsection{\tt{clozespace}}
\label{sec:command-clozespace}

\DescribeEnv{clozespace} |\begin{clozespace}|\oarg{options} \dots
\textit{some text} \dots |\end{clozespace}|: If you are using a
bigger font for the cloze text as for the normal text, you are getting
irregular distances between the lines:

\def\TmpOrbit{
\clozesetfont{\Huge\fontspec{Kalam}}
Today in the Discovery Lab we learned about three types of spacecraft
that are helping us explore \cloze{Mars}. The \cloze{spacecraft} are on
Mrs. Bratt’s Principal’s Reading Challenge board. One type of spacecraft
is the \cloze{orbiter}.
}

\begin{minted}{latex}
\clozesetfont{\Huge\fontspec{Kalam}}
Today in the Discovery ...
\end{minted}
\TmpReduceVspace
% https://en.wikipedia.org/wiki/Cloze_test#/media/File:Student_matching_cloze_terms_on_smartboard.jpg
\begin{TmpExample}
\TmpOrbit
\end{TmpExample}

\noindent
With the environment |clozespace| you are able to restore the regular
balanced line spacing. The default value for the option |spacing| is
\tt{\ClozeGetOption{spacing}}. Also take a look in the section about the
option |spacing| \TmpSecRef{sec:option-spacing}.

\begin{minted}{latex}
\begin{clozespace}[spacing=2]
...
\end{clozespace}
\end{minted}
\TmpReduceVspace
\begin{TmpExample}
\begin{clozespace}[spacing=2]
\TmpOrbit
\end{clozespace}
\TmpReduceVspace
\end{TmpExample}

\noindent
The environment |clozespace| uses the package
\href{https://www.ctan.org/pkg/setspace}{setspace} in the background for
setting the spacing between the lines.

%%
% \clozeline
%%

\subsubsection{\cmd{\clozeline}}
\label{sec:command-clozeline}

\DescribeMacro{\clozeline}
%
\cmd{\clozeline}\oarg{options}: To create a cloze line of a certain
width, use the command \cmd{\clozeline}. The default width of the line
is \tt{\ClozeGetOption{width}}. In combination with the other cloze
commands you can create for example an irregular alignment of the cloze
text.

\begin{minted}{latex}
Ut enim ad
\clozeline[width=1cm]\cloze{minim}\clozeline[width=3cm]
minim veniam
\end{minted}

\begin{TmpExample}
Ut enim ad \clozeline[width=1cm]\cloze{minim}\clozeline[width=3cm] minim
veniam,
\end{TmpExample}

%%
% \clozelinefil
%%

\subsubsection{\cmd{\clozelinefil}}
\label{sec:command-clozelinefil}

\DescribeMacro{\clozelinefil}
\cmd{\clozelinefil}\oarg{options}:
This command \cmd{\clozelinefil} fills the
complete available horizontal space with a line. Moreover,
\cmd{\clozelinefil} was used to create \cmd{\clozefil}.

\begin{TmpExample}
Lorem\clozelinefil
\end{TmpExample}

%%
% \clozestrike
%%

\subsubsection{\cmd{\clozestrike}}
\label{sec:command-clozestrike}

\DescribeMacro{\clozestrike}
\cmd{\clozestrike}\oarg{options}\marg{wrong text}\marg{correct text}:
This macro can be useful for worksheets that contain intentionally
errors. The pupils have to find and cross out the mistakes and
write the right solution on top of the errors.

\begin{minted}{latex}
Wolfgang Amadeus Mozart was born in \clozestrike{Vienna}{Salzburg}.
\end{minted}

\begin{TmpExample}
Wolfgang Amadeus Mozart was born in \clozestrike{Vienna}{Salzburg}.
\end{TmpExample}

\noindent
The option |line_color| has no effect on the lines produced by the
macro \cmd{\clozestrike}. The line color and and the text color
are both the same, because the pupils have to draw the lines.

\begin{minted}{latex}
Mozart’s father was called
\clozestrike[text_color=red]{Ludwig}{Leopold} Mozart.
\end{minted}

\begin{TmpExample}
Mozart’s father was called
\clozestrike[text_color=red]{Ludwig}{Leopold} Mozart.
\end{TmpExample}

%-----------------------------------------------------------------------
% Options
%-----------------------------------------------------------------------

\subsection{The options}

\subsubsection{Overview over all options}

\def\TmpDescriptionLine#1{
  {\footnotesize#1}
}

\begin{tabular}{lll}
\textbf{Option} &
\textbf{Description} &
\textbf{Reference} \\\hline

\tt{align} &
\TmpDescriptionLine{The alignment of a fixed-size cloze text (\cmd{\clozefix})}. &
\TmpSecRef{sec:option-align} \\

\tt{box_height} &
\TmpDescriptionLine{The height of a cloze box (\tt{clozebox}).} &
\TmpSecRef{sec:option-box-height} \\

\tt{box_rule} &
\TmpDescriptionLine{The thickness of the line around a \tt{clozebox}.} &
\TmpSecRef{sec:option-box-rule} \\

\tt{box_width} &
\TmpDescriptionLine{The width of a cloze box (\tt{clozebox}).} &
\TmpSecRef{sec:option-box-width} \\

\tt{debug} &
\TmpDescriptionLine{The debug or log level.} &
\TmpSecRef{sec:option-debug} \\

\tt{distance} &
\TmpDescriptionLine{The distance between the cloze text and the cloze line.} &
\TmpSecRef{sec:option-distance} \\

\tt{extend_count} &
\TmpDescriptionLine{The number of extension units (\cmd{\clozeextend}).} &
\TmpSecRef{sec:option-extend-count} \\

\tt{extend_height} &
\TmpDescriptionLine{The height of one extension unit (\cmd{\clozeextend}).} &
\TmpSecRef{sec:option-extend-height} \\

\tt{extend_width} &
\TmpDescriptionLine{The width of one extension unit (\cmd{\clozeextend}).} &
\TmpSecRef{sec:option-extend-width} \\

\tt{hide} &
\TmpDescriptionLine{Hide the cloze text.} &
\TmpSecRef{sec:option-hide} \\

\tt{font} &
\TmpDescriptionLine{The font of the cloze text.} &
\TmpSecRef{sec:option-font} \\

\tt{line_color} &
\TmpDescriptionLine{The color name to colorize the cloze line.} &
\TmpSecRef{sec:option-line-color} \\

\tt{margin} &
\TmpDescriptionLine{The additional margin between the normal and the cloze text.} &
\TmpSecRef{sec:option-margin} \\

\tt{min_lines} &
\TmpDescriptionLine{The minimum number of lines a \tt{clozepar} environment must have.} &
\TmpSecRef{sec:option-min-lines} \\

\tt{show} &
\TmpDescriptionLine{Show the cloze text.} &
\TmpSecRef{sec:option-show} \\

\tt{spacing} &
\TmpDescriptionLine{The spacing of the lines in the environment \tt{clozespace}.} &
\TmpSecRef{sec:option-spacing} \\

\tt{spread} &
\TmpDescriptionLine{The magnification or spreading factor of a gap.} &
\TmpSecRef{sec:option-spread} \\

\tt{text_color} &
\TmpDescriptionLine{The color name to colorize the cloze text.} &
\TmpSecRef{sec:option-text-color} \\

\tt{thickness} &
\TmpDescriptionLine{The thickness of a line.} &
\TmpSecRef{sec:option-thickness} \\

\tt{width} &
\TmpDescriptionLine{The width of a fixed size cloze (\cmd{\clozefix}).} &
\TmpSecRef{sec:option-show} \\

\tt{visibility} &
\TmpDescriptionLine{The visibility of the cloze text.} &
\TmpSecRef{sec:option-visibility} \\
\end{tabular}

%%
%
%%

\noindent
\begin{tikzpicture}[x=1cm,y=1cm,scale=0.8,transform shape]
% \draw (0,0) grid[step=1] (13,8);
% \draw[help lines] (0,0) grid[step=0.2] (13,8);
\node (cloze) at (-1,0) [node font=\fontsize{180}{1.2},anchor=south west]{
  \cloze[distance=15pt,thickness=3pt,margin=40pt]{cl…}
};

\node (distance) at (3,1.9) {distance};
\draw [|<->|] (4,1.7) --  (4,2.05);

\node (thickness) at (0,1.65) {thickness};
\draw [->] (1.3,2.2) --  (1.3,1.8);
\draw [->] (1.3,1) --  (1.3,1.4);

\node (margin) at (2.2,4) {margin};
\draw [|<->|] (1.5,3.5) --  (3,3.5);

\node (margin) at (11.75,4) {margin};
\draw [|<->|] (11,3.5) --  (12.5,3.5);

\node (text_color) at (4.5,6) {text_color};
\draw [->] (text_color) --  (6.4,5);

\node (line_color) at (7,0.5) {line_color};
\draw [->] (line_color) --  (7,1.5);
\end{tikzpicture}

%%
% clozefix
%%

\noindent
\begin{tikzpicture}[
  x=1cm,
  y=1cm,
  clozefix/.style={
    node font=\fontsize{40}{1.2}
  },
  scale=0.95,
  transform shape
]
% \draw (0,0) grid[step=1] (13,7);
% \draw[help lines] (0,0) grid[step=0.2] (13,7);

\node (right) at (7,6) [clozefix]{
  \clozefix[distance=10pt,thickness=2pt,width=8cm,align=right]{clozefix}
};

\node (center) at (5,4) [clozefix]{
  \clozefix[distance=10pt,thickness=2pt,width=8cm,align=center]{clozefix}
};

\node (left) at (3,2) [clozefix]{
  \clozefix[distance=10pt,thickness=2pt,width=8cm]{clozefix}
};

\node (width) at (5,0.7) {width};
\draw [|<->|] (0.8,1) --  (8.7,1);
\node (alignleft) at (6.5,2) {align=left};

\node (aligncenter) at (3.3,4) {align=center};

\node (alignright) at (7,6) {align=right};
\end{tikzpicture}
\clozereset

\begin{tikzpicture}
% \draw (0,0) grid[step=1] (12,7);
% \draw[help lines] (0,0) grid[step=0.2] (12,7);
\node (onlytikz) at (6,3.5) [draw,text width=3cm,scale=3,transform shape,thick] {
  \clozeparplain[min_lines=3,]{clozebox}
};

\node (box_height) at (0.5,3.5) [rotate=90] {box_height};
\draw [|<->|] (1,1.05) --  (1,5.95);

\node (box_width) at (6,0.5){box_width};
\draw [|<->|] (1.15,0.9) --  (10.85,0.9);
\end{tikzpicture}

%%
% Local and global options
%%

\subsubsection{Local and global options}

The \emph{cloze} package distinguishs between \emph{local} and
\emph{global} options. Besides the possiblity to set \emph{global}
options in the \cmd{\usepackage}\oarg{global options}\marg{cloze}
declaration, the cloze package offers a special command to set
\emph{global} options:
\cmd{\clozeset}\marg{global options}

%%
% \clozeset
%%

\subsubsection{\cmd{\clozeset}}
\label{sec:command-clozeset}

\DescribeMacro{\clozeset}
\cmd{\clozeset}\marg{global options}: The command can set \emph{global}
options for each paragraph.

\begin{minted}{latex}
\clozeset{text_color=red} Lorem \cloze{ipsum} dolor \par
\clozeset{text_color=green} Lorem \cloze{ipsum} dolor
\end{minted}

\begin{TmpExample}
\clozeset{text_color=red} Lorem \cloze{ipsum} dolor \par
\clozeset{text_color=green} Lorem \cloze{ipsum} dolor
\end{TmpExample}

\noindent \cmd{\clozeset} does not change the options within a
paragraph. As you can see in the example below the last \cmd{\clozeset}
applies the color green for both gaps.

\begin{minted}{latex}
\clozeset{text_color=red} Lorem \cloze{ipsum} dolor
\clozeset{text_color=green} Lorem \cloze{ipsum} dolor
\end{minted}

\begin{TmpExample}
\clozeset{text_color=red} Lorem \cloze{ipsum} dolor
\clozeset{text_color=green} Lorem \cloze{ipsum} dolor
\end{TmpExample}

\clozereset

%%
% \clozereset
%%

\subsubsection{\cmd{\clozereset}}
\label{sec:command-clozereset}

\DescribeMacro{\clozereset}
\cmd{\clozereset}: The command resets all \emph{global} options to the
default values. It has no effect on the \emph{local} options.

\begin{minted}{latex}
\clozeset{
  thickness=3mm,
  line_color=yellow,
  text_color=magenta,
  margin=-2pt
}
\end{minted}

\clozeset{thickness=3mm,line_color=yellow,text_color=magenta,margin=-2pt}

\begin{TmpExample}
Very \cloze{silly} global \cloze{options}.
\end{TmpExample}

\begin{minted}{latex}
\clozereset
\end{minted}
\clozereset

\begin{TmpExample}
\cloze{Relax!} We can reset \cloze{those} options.
\end{TmpExample}

%%
% \clozereset
%%

\subsubsection{\cmd{\clozeshow} and \cmd{\clozehide}}
\label{sec:command-clozeshow}
\label{sec:command-clozehide}

\DescribeMacro{\clozeshow} \DescribeMacro{\clozehide}
\cmd{\clozeshow} and \cmd{\clozehide}: This commands are shortcuts for
\cmd{\clozeset}\marg{show} and \cmd{\clozeset}\marg{hide}.

\begin{minted}{latex}
\clozehide
\end{minted}
\clozehide

\begin{TmpExample}
Lorem \cloze{ipsum dolor sit} amet, consectetur \cloze{adipisicing}
elit.
\end{TmpExample}

\begin{minted}{latex}
\clozeshow
\end{minted}
\clozeshow

\begin{TmpExample}
Lorem \cloze{ipsum dolor sit} amet, consectetur \cloze{adipisicing}
elit.
\end{TmpExample}

%%
% align
%%

\subsubsection{\tt{align}}
\label{sec:option-align}

\TmpOption{align}{left/center/right}:
%
Only the macro \cmd{\clozefix} \TmpSecRef{sec:command-clozefix} takes
the option \tt{align} into account. Possible values are \tt{left},
\tt{center} and \tt{right}. This option only makes sense, if the width
of the line is larger than the width of the text.

\newcommand{\TmpOptionsalign}[1]{%
  \noindent%
  \clozefix[align=#1,width=8cm]{Lorem ipsum}%
  \TmpDesc{#1}%
}

\begin{TmpExample}
\TmpOptionsalign{left}
\TmpOptionsalign{center}
\TmpOptionsalign{right}
\end{TmpExample}

%%
% box_height
%%

\subsubsection{\tt{box_height}}
\label{sec:option-box-height}

\TmpOption{box_height}{dimen}:
specifies the height of a cloze box. This option has only
an effect on the environment |clozebox| \TmpSecRef{sec:command-clozebox}.

\begin{minted}{latex}
\begin{clozebox}[box_height=5cm]
\end{minted}

\TmpReduceVspace

\begin{TmpExample}
\bigskip
\begin{center}
\begin{clozebox}[box_height=3cm,box_width=3cm]
box_height: 3cm; Lorem ipsum dolor sit amet ...
\end{clozebox}
\begin{clozebox}[box_height=2cm,box_width=3cm]
box_height: 2cm; Lorem ipsum dolor sit amet ...
\end{clozebox}
\begin{clozebox}[box_height=1cm,box_width=3cm]
box_height: 1cm; Lorem ipsum dolor sit amet ...
\end{clozebox}
\end{center}
\bigskip
\end{TmpExample}

%%
% box_rule
%%

\subsubsection{\tt{box_rule}}
\label{sec:option-box-rule}

\TmpOption{box_rule}{dimen}: specifies the thickness of the rule around a
cloze box. This option has only an effect on the environment |clozebox|
\TmpSecRef{sec:command-clozebox}.

\begin{minted}{latex}
\begin{clozebox}[box_rule=2pt]
\end{minted}

\TmpReduceVspace

\begin{TmpExample}
\bigskip
\begin{center}
\begin{clozebox}[box_rule=1pt,box_width=3cm]
box_rule: 1pt
\end{clozebox}
\begin{clozebox}[box_rule=2pt,box_width=3cm]
box_rule: 2pt
\end{clozebox}
\begin{clozebox}[box_rule=3pt,box_width=3cm]
box_rule: 3pt
\end{clozebox}
\end{center}
\bigskip
\end{TmpExample}

%%
% box_width
%%

\subsubsection{\tt{box_width}}
\label{sec:option-box-width}

\TmpOption{box_width}{dimen}:
specifies the width of a cloze box. This option has only an
effect on the environment |clozebox| \TmpSecRef{sec:command-clozebox}.

\begin{minted}{latex}
\begin{clozebox}[box_width=3cm]
\end{minted}

\TmpReduceVspace

\begin{TmpExample}
\bigskip
\begin{center}
\begin{clozebox}[box_width=2.5cm]
box_width: 2.5cm; Lorem ipsum dolor sit amet ...
\end{clozebox}
\begin{clozebox}[box_width=3cm]
box_width: 3cm; Lorem ipsum dolor sit amet ...
\end{clozebox}
\begin{clozebox}[box_width=5cm]
box_width: 4cm; Lorem ipsum dolor sit amet ...
\end{clozebox}
\end{center}
\bigskip
\end{TmpExample}

%%
% debug
%%

\subsubsection{\tt{debug}}
\label{sec:option-debug}

% Die Option |debug| definiert den Debug- oder Log-Level für den
% Lua-Code. Die folgenden Debug-Level können eingestellt werden.
The |debug| option sets the log level. The following debug levels can be
set.
% Es gibt einige Stellen im Lua-Code, an denen Print-Anweisungen
% eingebaut sind, die bei Bedarf Debug-Meldungen auf der Konsole
% ausgeben.
There are a few places in the Lua code where print statements are built
in, which output debug messages to the console if required. The default
value is \tt{\ClozeGetOption{debug}}.

\begin{itemize}
\item 1 (error)
\item 2 (warn)
\item 3 (info)
\item 4 (verbose)
\item 5 (debug)
\end{itemize}

%%
% distance
%%

\subsubsection{\tt{distance}}
\label{sec:option-distance}

\TmpOption{distance}{dimen}:
The option |distance| specifies the spacing between the baseline of the
text and the gap line. The larger the dimension of the option
|distance|, the more moves the line down. Negative values cause the line
to appear above the baseline. The default value is
\tt{\ClozeGetOption{distance}}.

\newcommand{\TmpOptiondistance}[1]{%
  \noindent%
  \clozefil[distance=#1]{Lorem ipsum dolor sit amet.}
  \TmpExpDesc{#1}
  \par%
}

\begin{TmpExample}
\TmpOptiondistance{1.5pt}
\TmpOptiondistance{3pt}
\TmpOptiondistance{-3pt}
\end{TmpExample}

\subsubsection{\tt{font}}
\label{sec:option-font}

\TmpOption{font}{cmd}: The option |font| is an alternative to the
commands \cmd{\clozefont} or \cmd{\clozesetfont}. If this option is used
\cmd{\clozefont} or \cmd{\clozesetfont} have no effect. The option can
used as a local or global option, but not as a package option.

\begin{minted}{latex}
Lorem \cloze[font=\Huge]{ipsum} dolor.
\end{minted}

\begin{TmpExample}
Lorem \cloze[font=\Huge]{ipsum} dolor.
\end{TmpExample}

%%
% hide and show
%%

\subsubsection{\tt{hide} and \tt{show}}
\label{sec:option-hide}
\label{sec:option-show}

\tt{[hide]} and \tt{[show]}:
By default the cloze text is displayed. Use the option |hide| to remove
the cloze text from the output.

\newcommand{\TmpOptionshow}[1]{%
  \noindent%
  Lorem ipsum \cloze[#1]{dolor sit amet}, consectetur
  \cloze[#1]{adipisicing} elit.%
  \TmpDesc{#1}%
}

\begin{TmpExample}
\TmpOptionshow{hide}
\TmpOptionshow{show}
\end{TmpExample}

%%
% line_color and text_color
%%

\subsubsection{\tt{line_color} and \tt{text_color}}
\label{sec:option-line-color}
\label{sec:option-text-color}

\TmpOption{line_color}{color name} and
\TmpOption{text_color}{color name}:
Values for both color options are color names.
To define your own color use the following command:

\begin{minted}{latex}
\definecolor{myclozecolor}{rgb}{0.1,0.4,0.6}
\cloze[text_color=myclozecolor]{Lorem ipsum}
\end{minted}
\definecolor{myclozecolor}{rgb}{0.1,0.4,0.6}

\newcommand{\TmpOptioncolor}[2]{%
  \noindent%
  \clozefil[#1=#2]{Lorem ipsum dolor sit amet, consectetur} %
  \TmpExpDesc{#2}%
  \par%
}

\begin{TmpExample}
\TmpOptioncolor{text_color}{myclozecolor}
\TmpOptioncolor{text_color}{red}
\TmpOptioncolor{text_color}{green}
\end{TmpExample}

\noindent You can use the same color names to colorize the cloze lines.

\begin{TmpExample}
\TmpOptioncolor{line_color}{myclozecolor}
\TmpOptioncolor{line_color}{red}
\TmpOptioncolor{line_color}{green}
\end{TmpExample}

\noindent
And now hide the clozes:

\clozehide
\begin{TmpExample}
\TmpOptioncolor{line_color}{myclozecolor}
\TmpOptioncolor{line_color}{red}
\TmpOptioncolor{line_color}{green}
\end{TmpExample}
\clozeshow


%%
% margin
%%

\subsubsection{\tt{margin}}
\label{sec:option-margin}

\TmpOption{margin}{dimen}:
The option |margin| indicates how far the line sticks up from the text.
The option can be used with the commands \cmd{\cloze}, \cmd{\clozefix}
and \cmd{\clozefil}. The default value of the option is
\tt{\ClozeGetOption{margin}}.

\newcommand{\TmpOptionmargin}[1]{%
  \noindent%
  Lorem ipsum \cloze[margin=#1]{dolor} sit amet.%
  \TmpDesc{#1}%
}

\begin{TmpExample}
\TmpOptionmargin{0pt}
\TmpOptionmargin{5mm}
\TmpOptionmargin{1cm}
\TmpOptionmargin{6em}
\TmpOptionmargin{-4pt}
\end{TmpExample}

% Folgt ein Satzzeichen direkt auf eine Lücke, so findet der
% Zeilenumbruch erst nach dem Satzzeichen statt. Auch ein noch so großer
% Wert für |margin| beeinflusst dieses Verhalten nicht.
\noindent Is a punctation mark placed directly after a gap, then the
line breaks after this punctation mark. Even the most large value of
|margin| does not affect this behavior.

\begin{TmpExample}
\clozeset{margin=3mm}
\cloze{Lorem}, \cloze{ipsum}. \cloze{dolor}; \cloze{sit}: \cloze{amet},
\cloze{consectetur}. \cloze{adipisicing}; \cloze{elit}: \cloze{sed},
\cloze{do}. \cloze{eiusmod}; \cloze{tempor}.
\end{TmpExample}

\clozereset

%%
% min_lines
%%

\newpage
\subsubsection{\tt{min_lines}}
\label{sec:option-min-lines}

\TmpOption{min_lines}{integer}: \tt{min_lines} stands for
“\textbf{min}imum \textbf{lines}”. With the option \tt{min_lines} you can
set how many lines a \tt{clozepar} at least must have. If the text to be
typeset results in fewer lines, empty lines are appended to the end. If
the text of a \tt{clozepar} produces more lines as specified in
\tt{min_lines}, the option has no effect at all. The option \tt{min_lines}
only affects the behavior of the environment \tt{clozepar}
\TmpSecRef{sec:command-clozepar} and the macro versions
\cmd{\clozeparplain} \TmpSecRef{sec:command-clozeparplain} and
\cmd{\clozeparcapture} \TmpSecRef{sec:command-clozeparcapture}.

% https://www.pinterest.de/pin/856176579144764867/

\begin{minted}{latex}
\clozeset{min_lines=4}
\begin{enumerate}
\item Battlements

\begin{clozepar}
Higher ...
\end{minted}

\clozeset{min_lines=4}

\def\TmpExampleMinlines{
\textbf{\Large{}Defending a Castle}

\noindent
Here is a list of features on castle in the Middle Ages that would help
to defend it.

\begin{enumerate}
\item Battlements

\begin{clozepar}
Higher than the walls and jutting out from them to allow
defenders to fire at people who had reached the walls
\end{clozepar}

\item Embrasures

\begin{clozepar}
Heavily protected entrance
\end{clozepar}

\item Portcullis

\begin{clozepar}
Arrow slits with a round hole for firing musket guns
\end{clozepar}
\end{enumerate}
}

\begin{multicols}{2}
\TmpHide
\begin{TmpExample}
\TmpExampleMinlines
\end{TmpExample}

\columnbreak

\TmpShow
\begin{TmpExample}
\TmpExampleMinlines
\end{TmpExample}
\end{multicols}

\clozereset

%%
% spacing
%%

\subsubsection{\tt{spacing}}
\label{sec:option-spacing}

\TmpOption{spacing}{number}:
%
This option provides support for setting the spacing between lines. A
larger font used for the cloze texts needs more line space to avoid
unsteady line distances. This option only affects the environment
|clozespace| \TmpSecRef{sec:command-clozespace}.

%%
% spread
%%

\subsubsection{\tt{spread}}
\label{sec:option-spread}

\TmpOption{spread}{number}:
%
% Vergrößere bzw. spreize eine Lücke um einen gewissen Faktor.
The option |spread| enlarges or spreads a gap by a certain factor.
% 0.5 beispielsweise vergrößert die Lücke um die Hälfte ihrer ursprünglichen bereite
$0.5$, for example, enlarges the gap by half of its original size.
% Ein spread-Faktor von 0 fügt keinen Abstand ein.
A spread factor of $0$ does not insert a space.
% Ein negativer spread-Faktor kann auch angewendet werden.
A negative spread factor can also be applied.
% Dadurch überlappen sich unter Umständen die Textzeichen.
This may cause the text characters to overlap.
% Diese Option kann nur in Verbindung mit den Kommandos cloze and clozenol eingesetzt werden.
This option can only be used in combination with the commands
\cmd{\cloze} \TmpSecRef{sec:command-cloze} and \cmd{\clozenol}
\TmpSecRef{sec:command-clozenol}.

\newcommand{\TmpOptionSpread}[1]{%
  \noindent%
  Lorem \cloze[spread=#1]{ipsum dolor sit} amet.%
  \TmpDesc{#1}%
}

\begin{TmpExample}
\TmpOptionSpread{-0.5}
\TmpOptionSpread{0}
\TmpOptionSpread{0.5}
\TmpOptionSpread{1}
\TmpOptionSpread{2}
\end{TmpExample}

\noindent
% Die Option spread fügt am Anfang und am Ende einer Lücke einen Abstand
% ein, der abhängig von der Breite des Texts in der Lücke ist.
The |spread| option inserts a space at the beginning and end of a gap
that depends on the width of the text in the gap.

\begin{TmpExample}
Lorem ipsum dolor \cloze[spread=0.5]{sit} amet. \TmpDesc{0.5}%
\noindent
Lorem ipsum \cloze[spread=0.5]{dolor sit} amet. \TmpDesc{0.5}%
\noindent
Lorem \cloze[spread=0.5]{ipsum dolor sit} amet. \TmpDesc{0.5}%
\noindent
\cloze[spread=0.5]{Lorem ipsum dolor sit} amet. \TmpDesc{0.5}%
\noindent
\cloze[spread=0.5]{Lorem ipsum dolor sit amet}. \TmpDesc{0.5}%
\end{TmpExample}

\noindent
% Zusätzlich zu der der Option spread kann auch noch die Option margin
% verwendet werden, die einen Abstand eingefügt werden, der unabhängig
% von der Breite der Lücke ist.
In addition to the |spread| option, the |margin| option can also be used,
which inserts a distance that is independent of the width of the gap.

%%
% thickness
%%

\subsubsection{\tt{thickness}}
\label{sec:option-thickness}

\TmpOption{thickness}{dimen}:
%
The option |thickness| indicates how thick the line is. The option
|distance| \TmpSecRef{sec:option-distance} is not affected by this
option, because the bottom of the line moves down. The default value of
this option is \tt{\ClozeGetOption{thickness}}.

\newcommand{\TmpOptionthickness}[1]{%
  \noindent%
  Lorem \cloze[thickness=#1]{ipsum dolor sit} amet.%
  \TmpDesc{#1}%
}

\begin{TmpExample}
\TmpOptionthickness{0.01pt}
\TmpOptionthickness{1pt}
\TmpOptionthickness{2pt}
\end{TmpExample}

%%
% width
%%

\subsubsection{\tt{width}}\label{sec:option-width}

\TmpOption{width}{dimen}:
%
The only command which can be changed by the option |width| is
\cmd{\clozefix} \TmpSecRef{sec:command-clozefix}. The default value of
the option is \tt{\ClozeGetOption{width}}.

\newcommand{\TmpOptionWidth}[1]{%
  \noindent%
  Lorem \clozefix[width=#1]{dolor} amet.%
  \TmpDesc{#1}%
}

\begin{TmpExample}
\TmpOptionWidth{3cm}
\TmpOptionWidth{5cm}
\TmpOptionWidth{7cm}
\end{TmpExample}

%%
% visibility
%%

\subsubsection{\tt{visibility}}\label{sec:option-visibility}

\TmpOption{visibility}{boolean}:
%
% Anstelle der entgegengesetzten Optionen |show| und |hide| können Sie
% die Option |visibility| verwenden und diese auf einen booleschen Wert
% setzen.
Instead of the opposite naked |show| and |hide| keys, you can use the
|visibility| option and set it to a boolean value.

\begin{minted}{latex}
Lorem \cloze[visibility=false]{ipsum} dolor.

Lorem \cloze[visibility=true]{ipsum} dolor.

Lorem \cloze[visibility=false]{ipsum} dolor.
\end{minted}

\def\TmpOptionVisibility#1{%
  \noindent%
  Lorem \cloze[visibility=#1]{ipsum} dolor.%
  \TmpDesc{#1}%
}

\begin{TmpExample}
\TmpOptionVisibility{false}
\TmpOptionVisibility{true}
\TmpOptionVisibility{false}
\end{TmpExample}
%

%-----------------------------------------------------------------------
% Handwriting fonts from CTAN and \TeX{} Live
%-----------------------------------------------------------------------

\subsection{Handwriting fonts from CTAN and \TeX{} Live}

If you want to imitate a hand-filled worksheet, then some handwriting
fonts are suitable for this purpose. This section is intended to provide
an overview of handwriting fonts available on CTAN and \TeX{} Live. The
fonts are listed in alphabetical order:

\def\TmpHandwritingExample{
  \begin{TmpExample}
  Lorem \cloze{ipsum} dolor sit amet, consetetur \cloze{sadipscing
  elitr, sed diam nonumy eirmod tempor} invidunt ut \cloze{labore et
  dolore magna aliquyam erat}, sed \cloze{diam} voluptua.
  \end{TmpExample}
}

\def\TmpHandwritingCtan#1#2{
  \subsubsection*{#1}

  \begin{tabbing}
    \hspace{2.8cm} \= \hspace{8cm} \kill

    CTAN: \>
    {\footnotesize\href{https://www.ctan.org/pkg/#2}{#2}}\\

    \TeX{} Live: \>
    \texttt{\footnotesize tlmgr install #2}\\

    Font selection: \>
    \texttt{\footnotesize\string\clozesetfont\{\string\fontspec\{#1\}\}}\\
  \end{tabbing}
  \vspace{-0.7cm}
  %

  \clozeset{margin=7pt}
  \clozesetfont{\Large\fontspec{#1}}
  \TmpHandwritingExample
}

% \TmpHandwritingCtan{LobsterTwo}{lobster2}
% \TmpHandwritingCtan{Miama Nueva}{miama}
\TmpHandwritingCtan{QT Brush Stroke}{qualitype}
\TmpHandwritingCtan{QT Florencia}{qualitype}
% \TmpHandwritingCtan{QT Handwriting}{qualitype}
\TmpHandwritingCtan{QT Linostroke}{qualitype}
\TmpHandwritingCtan{QT Merry Script}{qualitype}
\TmpHandwritingCtan{QT Slogantype}{qualitype}

%-----------------------------------------------------------------------
% Handwriting fonts from Google Fonts
%-----------------------------------------------------------------------

\subsection{Handwriting fonts from Google Fonts}

You can get many more free handwriting fonts from Google Fonts. This
section shows only a selection. I personally use the font named
\emph{Kalam} for my worksheets. All Google Fonts are available
in a \href{https://github.com/google/fonts}{Git respository}.

\begin{minted}{shell}
git clone https://github.com/google/fonts.git
\end{minted}

\noindent
The fonts are listed in alphabetical order:

\def\TmpHandwritingGoogle#1{
  \subsubsection*{#1}

  \begin{tabbing}
    \hspace{2.8cm} \= \hspace{8cm} \kill
    URL: \> {\footnotesize\url{https://fonts.google.com/specimen/\directlua{
      local font_name = '#1'
      font_name = font_name:gsub(' ', '+')
      tex.print(font_name)
    }}}
    \\

    Font selection: \>
    \texttt{\footnotesize\string\clozesetfont\{\string\fontspec\{#1\}\}}
    \\
  \end{tabbing}
  \vspace{-0.7cm}
  %
  \clozeset{margin=7pt}
  \clozesetfont{\Large\fontspec{#1}}
  \TmpHandwritingExample
}

%\TmpHandwritingGoogle{Amatic SC}
% \TmpHandwritingGoogle{Annie Use Your Telescope}
% \TmpHandwritingGoogle{Architects Daughter}
% \TmpHandwritingGoogle{Bad Script}
% \TmpHandwritingGoogle{Bubblegum Sans}
% \TmpHandwritingGoogle{Caveat Brush}
% \TmpHandwritingGoogle{Caveat}
% \TmpHandwritingGoogle{Cedarville Cursive}
\TmpHandwritingGoogle{Coming Soon}
% \TmpHandwritingGoogle{Covered By Your Grace}
% \TmpHandwritingGoogle{Crafty Girls}
% \TmpHandwritingGoogle{East Sea Dokdo}
% \TmpHandwritingGoogle{Fondamento}
% \TmpHandwritingGoogle{Ga Maamli}
% \TmpHandwritingGoogle{Gamja Flower}
% \TmpHandwritingGoogle{Give You Glory}
% \TmpHandwritingGoogle{Gloria Hallelujah}
\TmpHandwritingGoogle{Gluten}
% \TmpHandwritingGoogle{Gochi Hand}
% \TmpHandwritingGoogle{Grape Nuts}
% \TmpHandwritingGoogle{Hachi Maru Pop}
% \TmpHandwritingGoogle{Handlee}
% \TmpHandwritingGoogle{Homemade Apple}
% \TmpHandwritingGoogle{Indie Flower}
% \TmpHandwritingGoogle{Italianno}
% \TmpHandwritingGoogle{Julee}
% \TmpHandwritingGoogle{Just Another Hand}
% \TmpHandwritingGoogle{Just Me Again Down Here}
\TmpHandwritingGoogle{Kalam}
% \TmpHandwritingGoogle{Knewave}
% \TmpHandwritingGoogle{Kristi}
% \TmpHandwritingGoogle{La Belle Aurore}
% \TmpHandwritingGoogle{Lacquer}
% \TmpHandwritingGoogle{Leckerli One}
% \TmpHandwritingGoogle{Loved by the King}
% \TmpHandwritingGoogle{Ma Shan Zheng}
\TmpHandwritingGoogle{Mali}
% \TmpHandwritingGoogle{Mansalva}
% \TmpHandwritingGoogle{Marck Script}
% \TmpHandwritingGoogle{Meddon}
% \TmpHandwritingGoogle{Merienda}
% \TmpHandwritingGoogle{Montez}
% \TmpHandwritingGoogle{Moon Dance}
% \TmpHandwritingGoogle{Mr Dafoe}
% \TmpHandwritingGoogle{Mr De Haviland}
% \TmpHandwritingGoogle{Mrs Saint Delafield}
% \TmpHandwritingGoogle{Ms Madi}
% \TmpHandwritingGoogle{Nanum Brush Script}
% \TmpHandwritingGoogle{Neucha}
% \TmpHandwritingGoogle{Nothing You Could Do}
% \TmpHandwritingGoogle{Oooh Baby}
\TmpHandwritingGoogle{Oregano}
% \TmpHandwritingGoogle{Over the Rainbow}
% \TmpHandwritingGoogle{Pangolin}
% \TmpHandwritingGoogle{Patrick Hand SC}
% \TmpHandwritingGoogle{Patrick Hand}
% \TmpHandwritingGoogle{Permanent Marker}
% \TmpHandwritingGoogle{Playpen Sans}
% \TmpHandwritingGoogle{Playwrite Deutschland Grundschrift}
% \TmpHandwritingGoogle{Qwitcher Grypen}
% \TmpHandwritingGoogle{Rancho}
% \TmpHandwritingGoogle{Reenie Beanie}
% \TmpHandwritingGoogle{Rock Salt}
% \TmpHandwritingGoogle{Schoolbell}
% \TmpHandwritingGoogle{Seaweed Script}
% \TmpHandwritingGoogle{Sedgwick Ave}
% \TmpHandwritingGoogle{Shadows Into Light Two}
% \TmpHandwritingGoogle{Shadows Into Light}
% \TmpHandwritingGoogle{Shantell Sans}
% \TmpHandwritingGoogle{Short Stack}
% \TmpHandwritingGoogle{Sriracha}
% \TmpHandwritingGoogle{Sue Ellen Francisco}
% \TmpHandwritingGoogle{Sunshiney}
% \TmpHandwritingGoogle{Swanky and Moo Moo}
% \TmpHandwritingGoogle{The Girl Next Door}
% \TmpHandwritingGoogle{Vibur}
%\TmpHandwritingGoogle{Nanum Pen Script}
\clozereset

%-----------------------------------------------------------------------
% Special application areas
%-----------------------------------------------------------------------

\subsection{Special application areas}

This section lists examples that didn’t work in older versions of
the cloze package and required special treatment to work as expected.

\subsubsection{The math mode}

By default the package uses |\itshape| to format the cloze text. In
math mode you have to reset the cloze text format by calling
|\clozesetfont{}|. A known bug is: You can’t show and hide
a single display math formula. Only the last |\clozeshow| or
|\clozehide| takes effect on the whole document. Side note: The usage
of the \TeX{} primitive syntax |$$ $$| is not recommended.

\clozesetfont{}

%%
%
%%

\bigskip

\begin{multicols}{2}
\begin{minted}{latex}
\clozesetfont{}
$$1 + 1 = \cloze{2}$$
\clozesetfont{\itshape}
\end{minted}

\columnbreak

\begin{TmpExample}
$$1 + 1 = \cloze{2}$$
\end{TmpExample}
\end{multicols}

\bigskip

%%
%
%%

\noindent
|\[\]| should be used instead.

\begin{multicols}{2}
\begin{minted}{latex}
\[123 + 456 = \cloze{579}\]
\end{minted}

\columnbreak

\begin{TmpExample}
\[123 + 456 = \cloze{579}\]
\end{TmpExample}
\end{multicols}

%%
%
%%

\bigskip
\noindent
A cloze inside a display math environment should work fine:

\begin{multicols}{2}
\begin{minted}{latex}
\begin{displaymath}
2^{\cloze{2}} = 4
\end{displaymath}
\end{minted}

\columnbreak

\begin{TmpExample}
\begin{displaymath}
2^{\cloze{2}} = 4
\end{displaymath}
\end{TmpExample}
\end{multicols}

\bigskip

\noindent
The inline math mode works too:

\begin{minted}{latex}
${\sqrt[3]{\cloze{8}}} = 2$ and ${\sqrt[\cloze{3}]{\cloze{8}}} = 2$
\end{minted}

\begin{TmpExample}
\begin{center}
${\sqrt[3]{\cloze{8}}} = 2$ and ${\sqrt[\cloze{3}]{\cloze{8}}} = 2$
\end{center}
\end{TmpExample}

%%
% equation
%%

\begin{multicols}{2}
\begin{minted}{latex}
\begin{equation}
e^{\pi i} + 1 = \cloze{0}
\end{equation}
\end{minted}

\columnbreak

\begin{TmpExample}
\begin{equation}
e^{\pi i} + 1 = \cloze{0}
\end{equation}
\end{TmpExample}
\end{multicols}

%%
% amsmath: multline
%%

\noindent
The amsmath \texttt{multline} environment:

\begin{multicols}{2}
\begin{minted}{latex}
\begin{multline*}
p(x) = \cloze{590}x^4y^2 + \cloze{19}x^3y^3\\
- 12x^2y^4 - \cloze{12}xy^5
\end{multline*}
\end{minted}

\columnbreak

\begin{TmpExample}
\begin{multline*}
p(x) = \cloze{590}x^4y^2 + \cloze{19}x^3y^3\\
- 12x^2y^4 - \cloze{12}xy^5
\end{multline*}
\end{TmpExample}
\end{multicols}

%%
%
%%

\def\TmpSubtraction{
  Missing Subtraction Facts to 12

  \def\TmpMath##1{
    \hspace{0.3cm}\scalebox{1.5}{$##1$}
  }

  \clozeset{margin=8pt}
  \renewcommand{\labelenumi}{\alph{enumi})}
  \begin{enumerate}
  \item \TmpMath{8 - 6 = \cloze{2}}
  \item \TmpMath{10 - 4 = \cloze{6}}
  \item \TmpMath{7 - \cloze{2} = 5}
  \item \TmpMath{6 - \cloze{1} = 5}
  \item \TmpMath{\cloze{6} - 3 = 3}
  \item \TmpMath{\cloze{10} - 2 = 8}
  \end{enumerate}
}

\begin{multicols}{2}
\TmpHide
\begin{TmpExample}
\TmpSubtraction
\end{TmpExample}

\columnbreak

\TmpShow
\begin{TmpExample}
\TmpSubtraction
\end{TmpExample}

\end{multicols}

\clozesetfont{\itshape}
\clozereset

%%
% tabbing
%%

\subsubsection{The \tt{tabbing} environment}

\begin{minted}{latex}
\begin{tabbing}
col1 \hspace{1cm} \= col2 \hspace{1cm} \= col3 \hspace{1cm} \= col4 \\
\cloze{col1} \> \> \clozefix{col3} \\
\end{tabbing}
\end{minted}

\begin{TmpExample}
\begin{tabbing}
col1 \hspace{1cm} \= col2 \hspace{1cm} \= col3 \hspace{1cm} \= col4 \\
\cloze{col1} \> \> \clozefix{col3} \\
\end{tabbing}
\end{TmpExample}

%%
% picture
%%

\newpage
\subsubsection{The \tt{picture} environment}

\begin{minted}{latex}
\setlength{\unitlength}{0.8cm}
\begin{picture}(4.8,3.8)
\thicklines
\put(1,0.5){\line(2,1){3}} \put(4,2){\line(-2,1){2}} \put(2,3){\line(-2,-5){1}}

\put(0.4,0.2){\cloze{A}} \put(4.05,1.9){$B$} \put(1.8,3.1){$C$}

\put(3.1,2.5){$a$} \put(0.8,1.8){\cloze{b}} \put(2.5,0.9){\cloze{c}}
\end{picture}
\end{minted}

\def\TmpPicture{
  \setlength{\unitlength}{0.8cm}
  \begin{picture}(4.8,3.8)
  \thicklines
  \put(1,0.5){\line(2,1){3}}
  \put(4,2){\line(-2,1){2}}
  \put(2,3){\line(-2,-5){1}}
  \put(0.4,0.2){\cloze{A}}
  \put(4.05,1.9){$B$}
  \put(1.8,3.1){$C$}
  \put(3.1,2.5){$a$}
  \put(0.8,1.8){\cloze{b}}
  \put(2.5,0.9){\cloze{c}}
  \end{picture}
}
%

\begin{multicols}{2}
\TmpHide

\begin{TmpExample}
\center{\TmpPicture}
\end{TmpExample}

\columnbreak

\TmpShow

\begin{TmpExample}
\center{\TmpPicture}
\end{TmpExample}
\end{multicols}

\clozereset

%%
% tabular
%%

\subsubsection{The \tt{tabular} environment}

\begin{minted}{latex}
\begin{tabular}{l|l}
  \textbf{englisch} & \textbf{deutsch} \\\hline
  book & \clozefil{Buch} \\
  \clozefil{scissors} & Schere\\
  pen & \clozefil{Füller} \\
  \clozefil{pencil} & Bleistift\\
\end{tabular}
\end{minted}

\def\TmpTabluarEnglishGerman{
  \clozeset{distance=4pt}
  \renewcommand{\arraystretch}{1.5}
  \begin{tabular}{ll}
  \textbf{englisch} & \textbf{deutsch} \\
  \hline
  book & \clozefil{Buch} \\
  \clozefil{scissors} & Schere\\
  pen & \clozefil{Füller} \\
  \clozefil{pencil} & Bleistift\\
  \end{tabular}
  \vspace{0.2cm}
}

\begin{multicols}{2}
\TmpHide
\begin{TmpExample}
\center{\TmpTabluarEnglishGerman}
\end{TmpExample}

\columnbreak

\TmpShow
\begin{TmpExample}
\center{\TmpTabluarEnglishGerman}
\end{TmpExample}
\end{multicols}

%%
% forest
%%

\newpage
\subsubsection{The package \tt{forest}}

\begin{minted}{latex}
\begin{forest}
[Johann Sebastian \cloze{Bach}
  [Carl \cloze{Phillip} Emanuel
    [Johann Sebastian (der \cloze{Jüngere})]
  ]
  [Johann Christoph \cloze{Friedrich}
    [\cloze{Wilhelm} Friedrich Ernst]
  ]
  [\cloze{Johann Christian}]
]
\end{forest}
\end{minted}

\def\TmpForestBach{
  {
  \noindent
  \scriptsize
  \begin{forest}
  [Johann Sebastian \cloze{Bach}
    [Carl \cloze{Phillip} Emanuel
      [Johann Sebastian (der \cloze{Jüngere})]
    ]
    [Johann Christoph \cloze{Friedrich}
      [\cloze{Wilhelm} Friedrich Ernst]
    ]
    [\cloze{Johann Christian}]
  ]
  \end{forest}
  }
}

\TmpHide

\begin{TmpExample}
\begin{center}
\TmpForestBach
\end{center}
\end{TmpExample}

\TmpShow

\begin{TmpExample}
\begin{center}
\TmpForestBach
\end{center}
\end{TmpExample}

\clozereset

%%
% tikz
%%

\newpage
\subsubsection{The package \tt{tikz}}

%%
%
%%

\paragraph{Example Multiline alignment in TikZ}
\def\TmpTikzExampleAlignment{
  \begin{tikzpicture}[
    cloze box/.style={draw,text width=3cm},
    scale=0.6,
    transform shape
  ]
  \def\TmpLorem{
    \clozenol[margin=0pt]{Damit Ihr indess erkennt, woher
      dieser ganze Irrthum gekommen ist.}
  }

  \def\TmpTextBox##1##2##3{
    \node (##1) at (##2) [cloze box,align=##1] {\TmpLorem};
    \node [##3=0cm of ##1] {\texttt{\footnotesize{}align=##1}};
    \draw (##1) -- (tikz);
  }

  \node (tikz) at (4,4) [circle, text width=3cm, align=center] {
    \textbf{TikZ}\\Alignment of Multi-Line\\ Text
  };

  \TmpTextBox{left}         {0,8}{above}
  \TmpTextBox{center}       {4,8}{above}
  \TmpTextBox{right}        {8,8}{above}
  \TmpTextBox{flush right}  {8,4}{above}
  \TmpTextBox{none}         {8,0}{below}
  \TmpTextBox{flush center} {4,0}{below}
  \TmpTextBox{justify}      {0,0}{below}
  \TmpTextBox{flush left}   {0,4}{above}
  \end{tikzpicture}
}

\TmpHide
\begin{TmpExample}
\center{\TmpTikzExampleAlignment}
\end{TmpExample}

\TmpShow
\begin{TmpExample}
\center{\TmpTikzExampleAlignment}
\end{TmpExample}

%%
%
%%

\newpage
\paragraph{Example Timeline}

\def\TmpTimeline{\begin{tikzpicture}[
  scale=0.6,
  transform shape,
  text box/.style={black, text width=3cm, align=flush left, draw},
  arrow/.style={<-,thick,black,align=center},
  >=latex
]

  % limits
  \newcount\yearOne; \yearOne=1860
  \def\widthOfAxes{15}
  \def\numberOfDecades{11}
  \def\tenTickLength{0.40}
  \def\fiveTickLength{0.36}
  \def\oneTickLength{0.30}

  % help functions
  % ##1: year
  % ##2: distance arrow timeline: - below + above
  % ##3: arrow length
  % ##4: Text
  % ##5: left or right or empty (-> center)
  \def\yearArrowLabel(##1,##2,##3,##4,##5){
    \def\xy{{(##1-\yearOne)*\widthOfAxes/\numberOfDecades/10}};
    \pgfmathparse{int(##2*100)};
    \ifnum \pgfmathresult<0
      % below
      \def\yyp{{(\tenTickLength*(0.90+##2))}}; \def\yyw{{(\yyp-\tenTickLength*##3)}}
      \draw[arrow] (\xy,\yyp) -- (\xy,\yyw) node[below ##5,text box] at (\xy,\yyw) {\footnotesize\textbf{##1}: ##4};
    \else
      % above
      \def\yyp{{(\tenTickLength*(0.10+##2)}}; \def\yyw{{(\yyp+\tenTickLength*##3)}}
      \draw[arrow] (\xy,\yyp) -- (\xy,\yyw) node[above ##5,text box] at (\xy,\yyw) {\footnotesize\textbf{##1:} ##4};
    \fi
  }

  % axis
  \draw[->,thick] (-\widthOfAxes*0.03,0) -- (\widthOfAxes*1.03,0);

  % ticks
  \foreach \tick in {0,1,...,\numberOfDecades}{
    \def\x{{\tick*\widthOfAxes/\numberOfDecades}}
    \def\year{\the\numexpr \yearOne+\tick*10 \relax}
     % ten tick
    \draw[thick] (\x,\tenTickLength) -- (\x,-\tenTickLength) node[below] {\year};

    \ifnum \tick<\numberOfDecades
      % five tick
      \draw ({(\x+\widthOfAxes/\numberOfDecades/2)},0) --
            ({(\x+\widthOfAxes/\numberOfDecades/2)},\fiveTickLength);
      \foreach \ticko in {1,2,3,4,6,7,8,9}{
        \def\xo{{(\x+\ticko*\widthOfAxes/\numberOfDecades/10)}}
        % one tick
        \draw (\xo,0) -- (\xo,\oneTickLength);
      }
    \fi
  }

  % label
  \yearArrowLabel(1865,-1,3,
    Thirteenth \cloze{Amendment} to constitution abolishes \cloze{Slavery},)

  \yearArrowLabel(1875,1,2,
    \cloze{Civil Rights Act} is passed giving black citizens \cloze{equal} treatment.,)

  \yearArrowLabel(1896,-1,3,
    Plessy v. \cloze{Ferguson} say that segregation is \cloze{constitutional},)

  \yearArrowLabel(1909,1,2,
    \cloze{NAACP} is founded in New York and led by W.E.B. Du \cloze{Bois},)

  \yearArrowLabel(1946,1,2,
    U.S. Supreme court bans segregation on \cloze{Public Transit} (i.e. subways).,left)

  \yearArrowLabel(1955,-1,3,
    Rosa \cloze{Parks} refuses to give up her seat on a \cloze{bus} in \cloze{Alabama},left)

  \yearArrowLabel(1963,1,2,
    Martin Luther \cloze{King} Jr. delivers his ‘I Have a \cloze{Dream}’ speech in \cloze{Washington} D.C.,)

  \yearArrowLabel(1968,-3,2,
    Martin Luther King Jr. is \cloze{assassinated},right)
\end{tikzpicture}
}

\TmpHide

\begin{TmpExample}
\begin{center}
\TmpTimeline
\end{center}
\end{TmpExample}

\TmpShow

\begin{TmpExample}
\begin{center}
\TmpTimeline
\end{center}
\end{TmpExample}

\paragraph{Number pyramid}

\clozesetfont{\ttfamily}
\def\TmpNumberPyramid(#1,#2,#3,#4,#5,#6){
  \begin{tikzpicture}[
    node distance = 0pt,
    every node/.style = {
      draw,
      minimum width=10mm,
      minimum height=6mm,
      inner sep=0pt,
      outer sep=0pt,
      font=\ttfamily,
    }
  ]
  \node (n1)  {#1};
  %
  \node (n11) [below left=of n1.south]    {#2};
  \node (n12) [right=of n11]              {#3};
  %
  \node (n21) [below left=of n11.south]   {#4};
  \node (n22) [right=of n21]              {#5};
  \node (n23) [right=of n22]              {#6};
  \end{tikzpicture}
}

\def\TmpNumberPyramids{
  \TmpNumberPyramid(\clozenol{17},\clozenol{12},\clozenol{5},10,2,3)
  \TmpNumberPyramid(\clozenol{19},10,\clozenol{9},3,\clozenol{7},2)
  \TmpNumberPyramid(10,\clozenol{7},3,\clozenol{5},2,\clozenol{1})
}

\TmpHide

\begin{TmpExample}
\begin{center}
\TmpNumberPyramids
\end{center}
\end{TmpExample}

\TmpShow

\begin{TmpExample}
\begin{center}
\TmpNumberPyramids
\end{center}
\end{TmpExample}

\pagebreak
\PrintIndex
\end{document}
