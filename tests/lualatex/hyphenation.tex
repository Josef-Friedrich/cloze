%!TEX program = lualatex
\documentclass{article}
\input ../cloze-test-macros.tex
\usepackage{cloze}

\begin{document}
\TestTitleDesc{Hyphenation}{The automatic hyphenation should work.
At the moment it does not by Eltern}

\cloze{Wolfgang Amadeus Mozart} wurde am 27. \cloze{Jänner} 1756 um acht
Uhr abends in \cloze{Salzburg} in der \cloze{Getreidegasse} 9 in einer
\cloze{Dreizimmerwohnung} eines Mehrfamilienhauses
(Hagenauerhaus) geboren. Er war das \cloze{siebte} Kind seiner Eltern,
aber erst das zweite, das \cloze{überlebte}. Sein Vater war der aus
\cloze{Augsburg} zum Studium an der \cloze{Benediktineruniversität}
(1622–1810) \cloze{Salzburg} gezogene, \cloze{fürstbischöfliche
Kammermusikus} (ab 1757 \cloze{Hofkomponist} und ab 1763
\cloze{Vizekapellmeister}) Leopold Mozart, seine Mutter die in Sankt
Gilgen aufgewachsene Anna Maria Pertl. Schon am \cloze{Vormittag} nach
seiner Geburt wurde er im Dom von Salzburg auf die Namen \cloze{Joannes
Chrysostomus Wolfgangus Theophilus} getauft. Der Rufname war \cloze{Wolfgang,
Wolferl} oder auch \cloze{Woferl}.

Bereits im Alter von vier Jahren erhielten er und seine fünf Jahre
ältere Schwester \cloze{Maria Anna Walburga Ignatia}, genannt
\cloze{„Nannerl“}, vom Vater den ersten Musik- und
\cloze{allgemeinbildenden} Unterricht (\cloze{Klavier, Violine und
Komposition}). Schon 1761 zeichnete Vater Leopold ein Andante und ein
Allegro als des „\cloze{Wolfgangerl Compositiones}“ auf, denen ein
Allegro und ein Menuetto folgten, datiert auf den 11. bzw. 16. Dezember
1761. \cloze{Das fälschlicherweise immer wieder als früheste Komposition
genannte Menuett G-Dur mit einem Menuett C-Dur als Trio KV 1 entstand
vermutlich erst 1764. Auch Mozarts Begabung im Klavier- und Violinspiel
trat schnell hervor. 1762 folgten seine ersten Auftritte.}

Erste \cloze{Konzertreisen} Wolfgangs und seiner Schwester Nannerl mit
den \cloze{Elt\-ern} wurden Anfang 1762 nach \cloze{München} und
\cloze{Herbst} 1762 von Passau nach Wien arrangiert, um dem Adel die
talentierten Kinder zu präsentieren. Nach dem Erfolg der
\cloze{Wunderkind-Geschwister} in München und Wien startete die Familie
am 9. Juni 1763 zu einer ausgedehnten Tournee durch die deutschen Lande
und Westeuropa, die bis zur Rückkehr nach Salzburg am 29. November 1766
dreieinhalb Jahre dauerte. Stationen waren \cloze{München, Augsburg,
Ludwigsburg, Schwetzingen, Heidelberg, Mainz, Frankfurt am Main,
Koblenz, Köln, Aachen, Brüssel, Paris} (Ankunft am 18. November
1763),\cloze{Versailles, London (Ankunft am 23. April 1764), Dover,
Belgien, Den Haag, Amsterdam, Utrecht, Mechelen, erneut Paris} (Ankunft
10. Mai 1766), \cloze{Dijon, Lyon, Genf, Lausanne, Bern, Zürich,
Donaueschingen, Ulm und München, wo die Kinder bei Hofe oder in
öffentlichen Akademien musizierten}. Während dieser Reisen entstanden
die ersten Sonaten für Klavier und Violine sowie die erste Sinfonie
Es-Dur (KV 16). Die vier Sonaten für Klavier und Violine KV 6 bis 9 sind
1764 die ersten gedruckten Kompositionen Mozarts.

\end{document}
